\documentclass{article}
\usepackage{courier}
\renewcommand{\ttdefault}{pcr}
\usepackage[top=1in, bottom=1in, left=1in, right=1in]{geometry}
\usepackage{amsmath}
\usepackage{caption}
\usepackage{enumerate}
\usepackage{enumitem}
\usepackage{setspace}
\usepackage{amsmath}
\usepackage{parskip}
\usepackage{graphicx}
\usepackage{algorithm,algpseudocode}
\reversemarginpar% Keep \marginpar in left margin
\usepackage{amssymb}
\usepackage{amsmath}
\newcounter{parnum}
\newlength{\parnumwidth}
\setlength{\parnumwidth}{3em}
\newcommand{\N}{%
  \noindent\refstepcounter{parnum}%
  \makebox[0pt][r]{\makebox[\parnumwidth][l]{\textbf{\arabic{parnum}}}}%
  \hspace*{\parindent}\ignorespaces}
\setlength{\parindent}{0em}
\newcommand{\reset}{
	\setcounter{parnum}{0}}
\newcommand{\resetto}[1]{
	\setcounter{parnum}{#1}}

\newcommand{\transpose}[1]{$#1^T$}

\newcommand{\rn}{$\mathbb{R}^n$}
\newcommand{\Rm}{$\mathbb{R}^m$}
\newcommand{\rk}{$\mathbb{R}^k$}
\newcommand{\rsup}[1]{$\mathbb{R}^{#1}$}
\newcommand{\putcenter}[1]{$$ \text{#1} $$}
\newcommand{\xm}{$x^1, x^2, x^3 ... x^m$}
\newcommand{\xsu}[1]{$x^{#1}$}
\newcommand{\xud}[1]{$x_{#1}$}
\newcommand{\ysu}[1]{$y^{#1}$}
\newcommand{\musu}[1]{$\mu^{#1}$}
\newcommand{\s}[1]{$\sum_{i=1}^{#1}$}
\newcommand{\eps}{$\epsilon$}
\newcommand{\bran}[1]{\[  \text{#1} \]}
\newcommand{\anglebra}[1]{$ \left \langle \text{#1} \right \rangle$}
\newcommand{\anglebrat}[2]{
					$
					\left \langle
					  \begin{tabular}{c}
					  #1\\
					  #2\\
					  \end{tabular}
					\right \rangle
					$
		}

\newcommand{\anglebrathree}[3]{
					$
					\left \langle
					  \begin{tabular}{c}
					  \text{#1}\\
					  \text{#2}\\
					  \text{#3}\\
					  \end{tabular}
					\right \rangle
					$
		}
\newcommand{\evalat}[1]{$|_{\text{#1}}$}

\newcommand{\irange}[1]{ 1 \leq i \leq #1}
\newcommand{\range}[3]{ #1 \leq #2 \leq #3}
\newcommand{\rarrow}{$\Rightarrow$}
\newcommand{\inter}[1]{$\bigcap_{#1}$}
\newcommand{\uni}[1]{$\bigcup_{#1}$}
\newcommand{\Ga}{\Gamma}
\newcommand{\tran}[2]{$#1^T #2$}
\newcommand{\Lam}{\Lambda}
\newcommand{\Lamx}{$\Lambda_x$}
\newcommand{\lam}{\lambda}
\newcommand{\lamh}{$\hat{\lambda}$}
\newcommand{\vh}{$\hat{v}$}
\newcommand{\norm}[1]{$\|\text{#1}\|$}
\makeatletter
\newcommand*{\rom}[1]{\expandafter\@slowromancap\romannumeral #1@}
\makeatother
\newcommand{\Lagr}{$\mathcal{L}$}
\newcommand{\xb}{$\bar{x}$}
%\newcommand{\rb}{$\bar{r}$}
\newcommand{\xh}{$\hat{x}$}
\newcommand{\muh}{$\hat{\mu}$}
\newcommand{\minover}[1]{$\underset{#1}{\text{minimize}}$}
\newcommand{\maxover}[1]{$\underset{#1}{\text{maximize}}$}
\newcommand{\minx}{\minover{x \in X}}
\newcommand{\maxy}{\maxover{(x, \mu) \in Y}}
\newcommand{\curbra}[1]{\{ #1 \}}
\newcommand{\mub}{$\bar{\mu}$}
\newcommand{\nullset}{$\emptyset$}
\newcommand{\stl}{$<$}
\newcommand{\stg}{$>$}
\newcommand\tab[1][1cm]{\hspace*{#1}}
\newcommand{\gradx}{$\underset{x}{\nabla}$}
\newcommand{\grad}[1]{$\underset{\text{#1}}{\nabla}$}
\newcommand{\half}{$\frac{1}{2}$}
\newcommand{\prim}{$^{\prime}$}
\newcommand{\mmin}{$\in$}

\newcommand{\mas}[1]{\sum_{i=1}^{#1}}
\newcommand{\malam}[1]{\lambda_{#1}}
\newcommand{\matranspose}[1]{#1^T}
\newcommand{\marn}{\mathbb{R}^n}
\newcommand{\matran}[2]{#1^T #2}
\newcommand{\maLagr}{\mathcal{L}}
\newcommand{\mabrat}[2]{
					\[
					\left \langle
					  \begin{tabular}{c}
					  #1\\
					  #2\\
					  \end{tabular}
					\right \rangle
					\]
		}
\newcommand{\mabra}[1]{\[ #1 \]}
\newcommand{\magradx}{\underset{x}{\nabla}}
\newcommand{\mamat}[1]{$\begin{bmatrix}
                            #1
                          \end{bmatrix}$}

\newcounter{ALC@tempcntr}% Temporary counter for storage
\algnewcommand{\LeftComment}[1]{\Statex \(\triangleright\) #1}
\newcommand{\specialcell}[2][c]{%
  \begin{tabular}[#1]{@{}c@{}}#2\end{tabular}}



\begin{document}

\title{Data Integration on High-Difficulty Binary Classification}
\author{Julia Finch, Jesse Hellemn, Kentaro Hoffman, Zhe Zhang}
\maketitle


\section*{Abstract}

This paper explores the effectiveness of the data integration technique
Generalized Multiple Kernel Learning (GMKL) on high-difficuly binary
classification data. GMKL that integrates learned kernels from two disparate
data sources is systematically compared to performing GMKL on the two sources
naively concatenated as well as standard classification algorithms that are
performed on the concatenated sources. The variables that are systematically
varied are the number of informative dimensions in each source, the relative
information provided by each source, and the number of useless noise dimensions
added to each source. It is found that GMKL consistently outperforms its
competitors. This paper provides evidence that data integration techniques,
specifically GMKL, have the ability to drastically improve upon the performance
of naive concatenation.

\section*{Introduction}
For classification tasks in many fields, especially in medicine and the social
sciences, it is increasingly common for data to come from multiple disparate
data sources. For example, a sociologist might be interested in predicting
future income using two different sources, one on family environment and one on
school environment (we will use ``source" to refer to a single coherent
dataset). For the best analysis and classification results, all data sources
should be taken into account. However, most current machine learning
classification techniques have been developed for only single dataset inputs,
and it is not obvious how to best adapt these techniques to multi-source data.

A naive approach is to concatenate the sources together into one large feature
matrix, essentially treating all of the data as a single incoherent source.
Although simple, this method throws away
information about the separate sources and forces data analysis and
classification techniques to treat all of the sources in the same way. Various
data integration techniques have been proposed to more cleverly and effectively
combine multiple sources. Unfortunately, these techniques have been poorly
tested, and there has been no systematic evaluation of their effectiveness.

This paper evaluates one such specialized data integration technique,
Generalized Multiple Kernel Learning (GMKL), against traditional classifiers that
use concatenated data. We simulate many 2-source datasets with a variety of
properties for these comparative tests. We will use the term ``classifier" to
refer to both specialized data integration techniques such as GMKL as well as
naive methods that first concatenate all sources together.












\section*{Data Generation}

This paper follows the systematic philosophy of simulation studies advocated in
\cite{neto2014simulation}.


\subsection*{Criteria of Good Simulated Data}

In order to understand the behavior of the classifiers on 2-source data as
properties of the data are varied, we systematically created data to satisfy
all of the criteria detailed below. These criteria allow us to test all
the classifiers fairly against each other on consistent benchmarks.
\newline

\begin{minipage}{\textwidth}
\centering
\textbf{Criteria of Simulated Data}
\begin{enumerate}
    \item The dataset consists of two separate sources.
    \item \label{itm:separable} The two classes are separated by a true
        decision boundary that is known and calculable.
    \begin{itemize}
        \item or the two classes can be specified to overlap with percentage
            $p$, where $p=0$ leads to no overlap and $p=50$ leads to complete
            overlap
    \end{itemize}
    \item The decision boundary's complexity (and thus difficulty) can be
        parametrized and controlled.
    \item The reliability of each source can be specified.
    \item The noisiness of each source can be specified.
    \begin{itemize}
        \item \label{itm:noisy} extra meaningless dimensions can be
            added to each source
    \end{itemize}
\end{enumerate}
\label{tab:criteria}
\end{minipage}


\subsubsection*{Explanation for Data Criteria}

Suppose that a classifier $C$ only obtains 60\% classification accuracy on a
dataset $D$ (with datapoints evenly split amongst 2 classes). This could be
attributable to either:
\begin{itemize}
    \item The classifier is not well suited to certain properties of dataset
        $D$.
    \item Eighty percent of both classes overlap with each other. The best
        possible strategy in this area of overlap is to guess the class with
        50\% accuracy. An optimal classifier will then guess 40\% of the
        datapoints correctly and also classify the 20\% of non-overlapping
        datapoints perfectly.
\end{itemize}
Criterion \ref{itm:separable} ensures that the latter case does not occur,
so that classifier performance is attributable solely to its efficacy on
certain types of data.

Criteria \ref{itm:noisy} is more difficult than it at first seems. In
order to create a source with $N_{useful}$ useful dimensions and $N_{noisy}$
meaningless dimensions, we needed to both 1) make $N_{noisy}$ dimensions of
pure noise and 2) make $N_{useful}$ dimensions, all of which are always useful.
With a random coefficient linear model, it is impossible to verify that all
$N_{useful}$ dimensions are actually useful.



\subsection*{Data Generation Models}

\subsubsection*{Random Coefficient Linear Model}

A simple, common way to simulate data is to use a linear model with random
coefficients. This model generates data by:
\begin{enumerate}
    \item Specify the number of true latent variables $K$ along with the number
        of visible variables $N$
    \item Specify two distribution $D_j^0$ and $D_j^1$ of each latent variable
        $z_j$, $1 \leq j \leq K$, where $D_j^0$ is the distribution of $z_j$
        for negative classes and $D_j^1$ is the distribution of $z_j$ for
        positive classes
    \item Specify how each visible variable $x_j$ is generated from the latent
        variables $z_i$ with a formula of the form
        $$
        x_l
        = \sum_{i=1}^K \beta_i z_i
        + \sum_{i=1}^K\sum_{j=i}^K \beta_{i,j} z_i z_k
        + \text{higher-order-interactions}
        $$
        of linear combinations of arbitrary functions of the latent variables
    \item Specify every $\beta$ in the above formula
    \item For every datapoint
    \begin{enumerate}
        \item Pick which class the datapoint belongs to
        \item Sample each $z_j$ from its respective distribution for this class
        \item Generate each visible variable $x_j$ from its formula
    \end{enumerate}
\end{enumerate}

This model has significant shortcomings
\begin{itemize}
    \item It is not known how to systematically make the classification problem
        more or less difficult
    \item It is hard to know if the generated data overlaps
    \item It is hard to pick the $\beta$s to ensure that all of the criteria in
        \ref{tab:criteria} are satisfied
    \item There are many distributions and formulas to specify arbitrarily
    \item It is hard to know how many of the generated dimensions are useful
\end{itemize}





\subsubsection*{Feed-forward Network Model}

In order to satisfy all of the criteria in \ref{tab:criteria}, we created a
feed-forward conditional network (Figure \ref{fig:network_model}). This
network's process is given in detail Algorithm \ref{alg:network_model}. The
network model was inspired by Bayesian Networks; the output of each node is
sampled conditionally after that node's inputs ahave all been calculated. The
model is highly extensible. For this particular network, each sources
reliability, difficulty, noisiness, and relative amount of noise to useful
information could all be systematically varied independently of each other.
separately.

\begin{minipage}{\textwidth}
    \centering
    \includegraphics[scale=0.4]{network_model.png}
    \captionof{figure}{General schematic of the network data generation model.
        Every circle is a "node" with takes in an input from the previous layer
        and outputs a new layer. For example, $z_1$ will produce a
        $N$-dimensional XOR, where the parity depends on the input from $y$.}
    \label{fig:network_model}
\end{minipage}

\begin{algorithm}
\centering
\begin{algorithmic}[1]
    \item[] \LeftComment{Sample the $y$ layer}
    \State $y \leftarrow$ Bernoulli($p$)

    \item[]
    \item[] \LeftComment{Sample the $z$ layer}
    \ForAll{$z_i \in \{z_1, z_2\}$}
        \State $c \leftarrow$ Bernoulli($p_i$) \Comment{$p_i$ chance to corrupt
            source $i$}
        \State $z_i \leftarrow c * (1-y) + (1-c) * y$ \Comment{If corrupting,
            $z_i$ will be 0 if $y$ is 1 and 1 if $y$ is 0}
    \EndFor

    \item[]
    \item[] \LeftComment{Sample the $x$ layer of source 1, an $N_1$-dimensional XOR}
    \If{$z_1$ is even}
        \State $x^{(1)}_1 ... x^{(1)}_{N_1} \leftarrow N_1$-dimensional binary
        vector of even parity
    \Else
        \State $x^{(1)}_1 ... x^{(1)}_{N_1} \leftarrow N_1$-dimensional binary
        vector of odd parity
    \EndIf

    \item[]
    \item[] \LeftComment{Sample the $x$ layer of source 2, a $k_2$-period sine wave}
    \State $x^{(2)}_1 \leftarrow $ Uniform($-k_2\pi$, $k_2\pi$)
    \State $x^{(2)}_2 \leftarrow $ Uniform($-k_2\pi$, $k_2\pi$)
    \State $x^{(2)}_3 \leftarrow (x^{(2)}_1+x^{(2)}_2)sin(x^{(2)}_1) + mz_2$

    \item[]
    \item[] \LeftComment{Add noise}
    \State $x^{(1)} \leftarrow x^{(1)} + $ Normal(0, $\sigma_1$)
        \Comment{Gaussian noise added to each dimension independently for XOR sources}
    \State $x^{(2)}_3 \leftarrow x^{(2)}_3 + $ Normal(0, $\sigma_2$)
        \Comment{Gaussian noise only added in direction of margin for sine wave sources}
\end{algorithmic}
\caption{Data generation process for the network model}
\label{alg:network_model}
\end{algorithm}


The above model creates two sources of data with very different types of
decision boundaries.

The first source is an $N$-dimensional XOR, which consists of clusters of
points at every corner of an $N$-dimensional binary hypercube, where each
corner belongs to a different class than all of its $N-1$ closest neighboring
corners. When the input to $z_1$ is a 1 (representing a positive example), then
the output is sampled from all binary 0-1 $N$-dimensional vectors with an even
numbers of ones, also known as an even-parity XOR. Figure \ref{fig:xor} shows a
3D example. The complexity of the decision boundary is increases with the
number of clusters, and the number of clusters increases exponentially with the
dimension of the XOR. Noise is added to an XOR by adding sampling from
Normal(0, $\sigma_1$) independently for every dimension.

The second source is a sine wave in 3D space (Figure \ref{fig:sine_wave}); the
complexity of this decision boundary is directly proportional to the period of
the wave and inversely proportional to the margin $m$ between the two classes.
Noise is added to the sine wave only in the direction of the margin (points can
only be moved "up" or "down").

\begin{minipage}{\textwidth}
\begin{minipage}{.48\textwidth}
    \centering
    \includegraphics[width=\textwidth]{xor_3d_square.png}
    \captionof{figure}{A 3D-XOR with equal class size and $\sigma_1=0.2$ noise.
        The noise for the XORs created by this model are independent Normal(0,
        $\sigma_1)$ added to every dimension.}
    \label{fig:xor}
\end{minipage}
\hspace{.04\textwidth}
\begin{minipage}{.48\textwidth}
    \centering
    \includegraphics[width=\textwidth]{sine_wave_square.png}
    \captionof{figure}{A 3 period sine wave with $\sigma=0$ noise and a margin
        of $m=10$. The equation of this wave in $xyz$ coordinates is $z =
        (x+y)sine(x) + mc + Normal(0, \sigma_2)$, where $m$ is the margin and
        $c$ is the 1 or 0 class label.}
    \label{fig:sine_wave}
\end{minipage}
\end{minipage}

This model has several nice properties:
\begin{itemize}
    \item For both the $N$-dimensional XOR and the sine wave (with small enough
        margin $m$), every dimension is necessary for perfect classification.
    \item The complexity of the decision boundary for each source can be easily
        controlled.
    \item The reliability of each source can be varied independently of the
        other.
    \item The data can be made more noisy (to an extent) without
        compromising the separability of the two classes.
    \item Extra meaningless dimensions can easily be added to either source.
\end{itemize}

The reliability of each source is controlled by corrupting its input signal by
a speficied percentage. For example, to decrease the reliability of source 1 to
80\%, $p_1$ is set to 0.1. Then $z_1$ will only equal $y$ 90\% of the time, and
will give the wrong signal 10\% of the time. Note that 0\% reliability is
achieved when the signal is corrupted 50\% of the time, since the signal is a
binary 0 or 1.

The first experiments in this paper use a similar model to the one above,
except with another XOR as the second source instead of the sine wave. We refer
to this model as a ``Double XOR." The final experiment uses the exact model
described above.











\section*{Machine Learning Classifiers}

To evaluate the potential gain from intelligent data integration, we decided to
compare the classification accuracy of a kernel based integration technique
with commonly used vector concatenation benchmark algorithms.

\subsection*{Notation}
In describing different classification techniques, the following notation will
be used:
\begin{itemize}
\setlength\itemsep{0em}
\item (x, y) pair denotes a feature vector x \mmin \rsup{m} and its
    corresponding target value $\in \{0, 1\}$ .
\item \xud{j} denotes the $j$th feature in x.
\item there are N training examples and $i$th training pair is represented by
    (\xsu{i}, \ysu{i} ).
\end{itemize}


\subsection*{Vector Concatenation Benchmarks}

\subsubsection*{Gaussian Naive Bayes}
\textbf{Model:}
Gaussian Naive Bayes assumes that conditioned on the value of y, every feature
\xud{j} is generated independently i.e. $P(x|y) = \prod_{j=1}^{m} P(x_j|y)$. In
addition it also assumes that each $P(x_j|y) \sim \mathbf{N(\mu_{j,y},
    \sigma_y)}$ . To make $x$, the model would compute its posterior probability
$P(y|x)$ and pick the more likely case.

\textbf{Properties:}
The formulation is clean and simple. It is fast to train and is quite resistant
to extra noise dimensions. However, the excessively strong assumption of
independence means that we could only fit 2nd degree decision boundaries, so it
is expected to perform poorly on our highly complex Double XOR dataset.

\textbf{Implementation:}
We used sklearn's Gaussian Naive Bayes classifier. No hyper-parameter tuning is
required.

\subsubsection*{Random Forest}
\textbf{Model:}
Random Forest uses an ensemble of decision trees to make a prediction for $x$.
Each decision tree is based by a bootstrap sample of the training data and the
splitting nodes are constrained to a random subset of all features. In this
manner, the over-fit of decision tree could be controlled.

\textbf{Properties:}
Random Forest is known to be very resistant to extra noise dimensions since
uninformative feature would never be used for a split in a decision tree.

\textbf{Implementation:}
We used cross validation to select the optimal number of trees $\bar{t} \in
\{10, 100\}$.

\subsubsection*{K-Nearest Neighbors}
\textbf{Model:}
KNN makes a prediction on an unknown data-point by the majority voting of the
K-closest points i.e. $$f(x) = sign(\sum_{x^i \in \text{K closest points}} y^i)
$$

\textbf{Properties:}
KNN is able to fit extremely complex decision boundaries. However, it often
suffers from the curse of high dimensionality. Moreover, it
could not discriminate against noise dimensions since they are incorporated as
part of the Euclidean distance.

\textbf{Implementation:}
We used cross validation to select the best number of neighbors $\bar{k} \in
\{1, 2, 10\}$




\subsubsection*{RBF Support Vector Machine}
\textbf{Model:}
Similar to KNN, Support Vector Machine makes recommendation by considering the
weighted voting of a set of similar data points
$$
f(x)
= sign(\sum_{i=1}^{N} y^i d^i K(x, x^i))\ \text{where K(x, $x^i$)}
= exp(-\gamma|x - x^i|^2)
$$
where $d^i$ is learned from data by finding the largest margin separating
hyperplane in the projected space.

\textbf{Properties:}
SVM can fit complex decision boundaries. Moreover, it does not suffer from the
curse high dimensionality to the extent that KNN does. These two
characteristics make SVM the go-to algorithm for classification. However, just
like KNN, it is unable to effectively discriminate against extra noise
dimensions.

\textbf{Implementation:}
Cross-validation is performed to search for the best regularization parameter
$\bar{C} \in \{0.1, 1, 10, 100\}$ and $\bar{\gamma} \in \{.01, .1, 1, 10\}$




\subsection*{Generalized Multiple Kernel Learning (GMKL)}
GMKL learns separate kernels for each data source before integrating the
separate kernels in an optimization step that's based on a global error
measure; the resulting combined kernel is then fed into an SVM. Although this
method uses a SVM for actual classification, the separate kernels allow each
source to have its own representation. A brief description of the algorithm is
described below, with a general flowchart of the process shown in Figure
\ref{fig:implementation_flowchart} and a more detailed explanation of the
optimization part in Algorithm \ref{alg:gmkl}. A full explanation of the GMKL
algorithm can be found in \cite{gmkl}.

\textbf{Model:}
The final trained model is almost the same as those from SVM, except that the
final kernel function is actually learned from data. More specifically, the
final kernel is a convex combination of a set of predefined kernel functions:
$$K(x, x^i) = \sum_{q=1}^{Q} \theta_q * K_q(x, x^i) \text{where }\theta_q
\text{is learned from data}.$$

\textbf{Formulation as Optimization Problem}
\begin{align*}
&\text{\minover{\theta, v, b}} & &C  \frac{1}{N} \sum_{i=1}^{N} L(f_{\theta, w, b}(x^i), y^i) + 1/2 \sum_{q=1}^{Q}|\frac{v_q|^2}{\theta_q} \\
&\text{subject to} & &\beta|\theta|_2^2 + (1-\beta){|\theta|_1} \leq 1
\end{align*}


Where C is the regularization parameter, $\beta$ is the elastic net parameter,
L is the hinge loss function and $$f_{\theta, w, b}(x^i) = \sum_{q=1}^{Q} w_q
\phi_q(x^i) \sqrt{\theta_q} + b $$

Now if we define v by $v_q := \frac{w_q}{\sqrt{\theta_q}}$, then we have a
convex optimization problem:

\begin{align*}
&\text{\minover{\theta, v, b}} & &C  \frac{1}{N} \sum_{i=1}^{N} L(f_{\theta, w, b}(x^i), y^i) + 1/2 |w|^2 \\
&\text{subject to} & &\beta|\theta|_2^2 + (1-\beta){|\theta|_1} \leq 1 \\
&\text{Where}  & &f_{\theta, v, b}(x^i) = \sum_{q=1}^{Q} v_q \phi_q(x^i)  + b
\end{align*}

Observe that the elastic net constraint ensures sparsity and grouping effect
for the set of kernels chosen for the final model.

\textbf{Optimization Algorithm}

\begin{algorithm}
\centering
\begin{algorithmic}[1]
    \item[] \LeftComment{Initialization}
    \State $t \leftarrow 0$
    \State Let $\theta$ be uniformly initialized subject to the elastic net constraint

    \item[]
    \While{difference between the upper bound and lower bound is less than \eps}
        \State $\alpha^{t} = \underset{\alpha}{argmax}D(\theta^{t-1}, \alpha)$
            \Comment Solve dual problem
        \State $h^t(\theta) = \underset{1 \leq i \leq t}{max} D(\theta, \alpha^i)$
            \Comment Construct a cutting plane model
        \State Calculate a lower bound and an upper bound for the optimal solution $\overline{D_t}, \underline{D_t}$ and an improvement set level set $L = \{\theta: h^t(\theta) \leq \text{some convex combination of }\overline{D_t}, \underline{D_t}\}$
        \State Project $\theta^{t-1}$ to L to obtain $\theta^t$
    \EndWhile
\end{algorithmic}
\caption{Level method for the MKL[1]}
\label{alg:gmkl}
\end{algorithm}

\textbf{Implementation}
For each `independent' data source, we constructed 10 RBF kernels with the
width $\gamma \in \{2^{-3}, 2^{-2}, 2^{-1} ..., 2^{6}\}$. Then we trained our
model with the regularization parameter $C$ fixed to 100.

\begin{minipage}{\textwidth}
    \centering
    \includegraphics[scale=.4]{implementation_flowchart.png}
    \captionof{figure}{Flow Chart of Our Model}
    \label{fig:implementation_flowchart}
\end{minipage}

\textbf{Concatenated GMKL}
To isolate the effect of data integration, we also applied GMKL to concatenated
data for comparison. More specifically, we constructed only 10 RBF kernels with
$\gamma \in \{2^{-3}, 2^{-2}, 2^{-1} ..., 2^{6}\}$ for the concatenated dataset
and then trained GMKL algorithm with C fixed at 100.\\













\section*{Experiments}



\subsection*{Experiment 1: Data Dimension Scaling}
The first experiment used the network data generation model shown in Figure
\ref{fig:network_model}, using an XOR for both sources. The two sources always
had the same dimension as each other. This first experiment tested GMKL,
Concatenated GMKL, SVM, KNN, and Random Forest as the dimensionality of the XOR
(and thus the difficulty of the classification task) increased. This test
increases to a maximum of 7 dimensions per source, and so at first glance this
may seem like a very small problem. But recall that the XOR has a very complex
decision boundary, as evidenced by every classifier's poor performance at even
this low XOR dimensionality.

\subsubsection*{Experiment 1 Parameters}
\begin{center}
\begin{tabular}{|c|c|c|c|c|c|c|c|}
\hline
$M_{i,useful}$ & $M_{i, noisy}$ & $N$ & T &  $p_1, p_2$ & $\sigma_i$ & C &  k  \\
\hline
[2,3,4,5,6,7] & 0 & 5000 & 1:2 & 0,0 & 0.2 & 1:1 & NA  \\
\hline
\end{tabular}
\end{center}



\begin{minipage}{\textwidth}
\centering
\includegraphics[scale=0.4]{experimentpic1.png}
\captionof{figure}{An illustration of how resistant GMKL is to the affect of an
    increasingly complex XOR decision boundary compared to the standard
    classifiers.}
\label{fig:exp_1}
\end{minipage}


From Figure \ref{fig:exp_1} we can see that as the number of XOR dimensions
increases, the classification accuracy of all of the classifiers decreases.
This makes sense because as the XOR dimension increases, the
decision boundary becomes more complex causing classification to become more
difficult. However, while all classifiers have decreasing accuracy, we can see
that GMKL performs much better than the other classifiers with nearly a 35
percent better classification accuracy compared to the classical SVM at five
dimensions. In fact, there is a very intriguing pattern on display here as the
the GMKL at $2n$ dimensions seems to be performing about as well as the
classical SVM at $n$ dimensions. This seems to indicate that doing
classification on the data sources as separate entities is highly desirable for
complicated classification problems.











\subsection*{Experiment 2: Corrupted XOR Sources}

Another variable that we decided to vary is the relative importance of the
disparate sources. The purpose of this experiment is to evaluate how the
various classifiers perform when one source is more important than the other.
We want to see which classifiers are able to identify the important
sources/features.

\subsubsection*{Experiment 2 Parameters}
\begin{center}
\begin{tabular}{|c|c|c|c|c|c|c|c|}
\hline
$M_{i,useful}$ & $M_{i, noisy}$ & $N$ & T &  $p_1, p_2$ & $\sigma_i$ & C &  k  \\
\hline
[3,5,7] & 0 & 5000 & 1:2 & [0.0,0.1,0.2,0.3,0.4,0.5],[0.0,0.1,0.2,0.3,0.4,0.5] & 0.2 & 1:1 & NA  \\
\hline
\end{tabular}
\end{center}



In order to test how the classifiers perform with different level of corruption
on the Double XOR data, we varied the probability of corruption for each XOR
source in increments of 0.1 from 0 to 0.5. We performed this experiment three
times, holding the dimension of each source static at three, five, and seven.
The results of this experiment for GMKL and Concatenated GMKL performed on
five dimensional XOR are displayed in the figures below. See the appendix for
the results from all classifiers tested on five dimensional XOR.


\begin{minipage}{\textwidth}
\begin{minipage}{.48\textwidth}
    \centering
    \includegraphics[width=\textwidth]{dxor_heat_gmkl.png}
    \captionof{figure}{An illustration of the ability of GMKL to ignore useless
        data sources.}
    \label{fig:dxor_heat_gmkl}
\end{minipage}
\hspace{.04\textwidth}
\begin{minipage}{.48\textwidth}
    \centering
    \includegraphics[width=\textwidth]{dxor_heat_conc.png}
    \captionof{figure}{An illustration that demonstrates Concatenated GMKL is
        vulnerable to unimportant sources.}
    \label{fig:dxor_heat_conc}
\end{minipage}
\end{minipage}




We see that the results of Experiment 2 are consistent with the results of
Experiment 1 in that GMKL consistently outperforms Concatenated GMKL. This indicates that the data integration part of GMKL is extremely useful as running GMKL on the concatenated data does not provide nearly as good classification accuracy as the GMKL that treats sources separately. We also see that GMKL is the
least affected by one of the two sources becoming corrupted. This is evidence
that GMKL effectively identifies which sources are useful and relies primarily
on those sources.



\subsection*{Experiment 3: Noise Dimensions}

Another variable to explore is the number of noise dimensions. This experiment
explores how each classifier is able to handle additional dimensions that do
not provide useful information regarding the class each datapoint belongs to.
Ideally, the classifiers would be able to identify them as noise dimensions and
ignore these dimensions in their predictive model.

For this experiment, we added the same number of noise dimensions to each XOR
source. We performed this experiment three times, when there were three, five,
and seven XOR dimensions per source. We varied the proportion of dimensions that were
noise dimensions. The proportion varied from no noise dimensions, to a quarter,
third, half, and then finally two thirds noise dimensions. The purpose of
measuring the proportion of noise variables as opposed to the number of noise
variables is so that the three experiments over different XOR dimensions can be
accurately compared.


\subsubsection*{Experiment 3 Parameters}
\begin{center}
\begin{tabular}{|c|c|c|c|c|c|c|c|}
\hline
$M_{i,useful}$ & $M_{i, noisy}$ & $N$ & T &  $p_1, p_2$ & $\sigma_i$ & C &  k  \\
\hline
3& [0,1,2,3,6] & 5000 &1:2 & 0,0 & 0.2 & 1:1 & NA  \\
\hline
5& [0,1,3,5,10] & 5000 &1:2 & 0,0 & 0.2 & 1:1 & NA  \\
\hline
7& [0,1,4,7,14] & 5000 &1:2 & 0,0 & 0.2 & 1:1 & NA  \\
\hline
\end{tabular}
\end{center}



The results from the three dimensional Double XOR experiment are represented in
Figure \ref{fig:noise_dim_line}. The results from the five dimensional and
seven dimensional Double XOR can be found in the appendix.

\begin{minipage}{\textwidth}
    \centering
    \includegraphics[scale=0.7]{Noise_Dim_line.png}
    \captionof{figure}{An illustration of how resistant GMKL is to the affect of noise dimensions compared to the standard classifiers.}
    \label{fig:noise_dim_line}
\end{minipage}

The purple line in the plot above shows how noise resistant GMKL is while the
accuracy of Concatenated GMKL falls when only a single noise dimension is
added. When there are the same number of noise dimensions as there are useful
dimensions, the accuracy of GMKL is still greater than 95 percent. This tells
us that GMKL effectively ignores noise dimensions. It is worthy to note the
Random Forest classifier is considered noise resistant and is even less
affected by noise than GMKL. However, GMKL is initially so much more accurate
than Random Forest, that as far as we tested, the accuracy of GMKL remains
superior to the Random Forest classifier.



\subsection*{Experiment 4: Corrupted Sine and XOR Sources}

This experiment follows the same structure as Experiment 2 in terms of varying
the corruption probabilities to each source. The difference is that instead of
using two sources with the XOR structure, one XOR source was used and one Sine
source was used. We also varied the corruption probabilities in different
increments. The corruption probability on each source ranged from 0 to 0.45 in
increments of 0.15. We executed this experiment on three dimensional XOR and
Sine data with a period of two. The results are displayed below.


\subsection*{Experiment 4 Parameters}
\begin{center}
\begin{tabular}{|c|c|c|c|c|c|c|c|}
\hline
$M_{i,useful}$ & $M_{i, noisy}$ & $N$ & T &  $p_1, p_2$ & $\sigma_i$ & C &  k  \\
\hline
3 & 0 & 0 & 1:2 & [0.0,0.15,0.3,0.45],[0.0,0.15,0.3,0.45] & 0.2 & 1:1 & [1,2,3]  \\
\hline
\end{tabular}
\end{center}


\begin{minipage}{\textwidth}
    \centering
    \includegraphics[width=\textwidth]{SineXORCorrupt.png}
    \captionof{figure}{An illustration that shows GMKL is unable to classify
        accurately using solely a Sine source.}
    \label{fig:sinexor_heatmaps}
\end{minipage}


Again we can see that GMKL outperforms Concatenated GMKL overall. From the heat
maps above, we can observe that GMKL does not handle Sine data well. When the
XOR source is heavily corrupted, the prediction accuracy of GMKL falls to
barely better than a coin flip. On the opposite end of the heat map, we can
note that GMKL performs very well when the Sine source is heavily corrupted as
long as the XOR source remains intact. This means that GMKL is insensitive to
unreliable data. GMKL is successfully able to identify the source that does not
contribute to classification accuracy and ignore it.

\section*{Conclusion}
From these experiments we have seen not only the usefulness of GMKL, but also
the issues that traditional classifiers face in data integration. Highly
complex decision boundaries, variably useful data sources, and noisy dimensions have all been shown to negatively affect the
classification accuracy of traditional classifiers such as KNN, SVM and Random
Forest. Although these aspects negatively affect the accuracy of GMKL as well, we saw that GMKL was much more resistant to these factors than the standard classifiers. Not only does this illustrate the usefulness of GMKL as a tool for
classification and data integration, but the way in which GMKL generates
kernels separately indicates the importance of treating your data sources as
separate entities and the damage you will do to your classification accuracy if
you concatenate it all together without any thought. The drastic improvements
seen by using an intelligent data integration technique over naive
concatenation illustrates that data integration is extremely useful and is
worth developing.


\section*{Future Work}
Data integration, data simulation, and benchmarking of machine learning
techniques are all fields in need of further investigation. In particular, the
behavior of machine learning and data integration techniques as functions of
properties of the data and its decision boundaries is very poorly understood.
Some fruitful further questions for research include:

\begin{itemize}
    \item \textbf{Shared Information Across Sources:} An important component of
        real world multi-source datasets is that information is shared across
        sources. That is, the sources aren't completely independent of each
        other, but both give insights into similar underlying hidden variables.
        The current data generation network model presented above does not have
        a way to systematically vary the information that is shared between
        both sources, but it would be very interesting to investigate if
        and how this property of the data affected the performance of
        classifiers.
    \item \textbf{Neural Networks:} Artificial neural networks are very powerful
        classifiers that are growing in popularity. They are an extremely
        extensible and flexible framework, with specialized versions existing
        for specific problems within the machine learning. They were not
        included in these experiments because they are too flexible. Neural
        networks have many more parameters than most popular classifiers, so it
        was not known how to pick an appropriate network architecture and
        topology for comparison against the other classifiers.
    \item \textbf{Theoretical Framework of Decision Boundaries:} We saw in
        Experiment 4 that GMKL performed well on the XOR dataset and not well
        on the Sine dataset. We would like to develop a theoretical framework
        that characterizes the aspects of decision boundaries that enable or
        disable a particular data integration technique from performing well on
        a dataset. We would like to come up with a concrete method for
        determining the best choice of data integration technique for a
        specific decision boundary.
    \item \textbf{More Computationally Expensive Testing:} All of the
        experiments in this paper ran with $N=5000$ data points. For really
        thorough results, it'd be better to run these experiments for several
        values of $N \in \{1000, 5000, 1000, 2000, 50000, 10000\}$ as well as
        for more values of noise, more dimensions, more corruption levels, more
        width $\gamma$ and regularization $C$ of the SVMs, etc.
    \item \textbf{More varied simulated data:} This paper only used two types
        of decision boundaries, that of the sine wave and that of the
        $N$-dimensional XOR. There are many types of decision boundaries that
        are not represented here and many possible properties of data that
        aren't present here. Possible unimplemented ideas were to create
        separable data in $m < N$ dimensions and then to use (mostly) monotonic
        transformations to map the $m$ dimensions into $N$-dimensional space,
        to create $N$ dimensional separable data.
\end{itemize}






\newpage
\section*{Appendix}


\begin{center}
\begin{tabular}{|c|c|}
\hline
\textbf{Variable} &\textbf{ Description}\\
\hline
\hline
$M_{i,useful}$ & Number of useful Data Dimensions for Data Source i\\
\hline
$M_{i, noisy}$ & Number of Noisy Dimensions for Data Source i\\
\hline
$N$ & Number of Data Points Generated (training and testing)\\
\hline
T & Ratio of Training to Testing \\
\hline
$p_1, p_2$ & Probability of Data Source Corruption\\
\hline
$\sigma_i$ & Variance of Gaussian Noise \\
\hline
C & Proportion of Data points in each class \\
\hline
k & Period of Sine Curve\\
\hline
\end{tabular}
\end{center}

\subsection*{Experiment 1: Data Dimension Scaling}

\begin{minipage}{\textwidth}
\centering
\begin{tabular}{|c| c| c| c| c| c|}
\hline
XOR Dimensions (per source) & KNN & Random Forest & SVM & Concatenate + GMKL & GMKL \\
\hline
2 & 99.0 & 96.6 & 99.0 & 99.9 & 99.9 \\
\hline
3 & 86.2 & 79.7 & 83.6 & 97.5 & 99.8 \\
\hline
4 & 70.1 & 68.7 & 68.9 & 78.6 & 97.5 \\
\hline
5 & 60.8 & 58.1 & 61.4 & 66.2 & 96.4 \\
\hline
6 & 57.8 & 55.0 & 60.2 & 64.6 & 83.7 \\
\hline
7 & 58.1 & 54.8 & 57.2 & 63.5 & 74.9 \\
\hline
\end{tabular}
\end{minipage}

\subsection*{Experiment 2: Corrupted XOR Sources}

\begin{minipage}{\textwidth}
\centering
\begin{tabular}{|c| c| c| c| c| c| c|}
\hline
$p_1$ & $p_2$ & KNN & Random Forest & SVM & Concatenate + GMKL & GMKL \\
\hline
0.0 & 0.0 & 61.5 & 58.6 & 62.6 & 67.7 & 94.7 \\
\hline
0.0 & 0.1 & 61.2 & 56.8 & 62.0 & 65.8 & 92.3 \\
\hline
0.0 & 0.2 & 57.6 & 58.3 & 59.5 & 60.4 & 88.8 \\
\hline
0.0 & 0.3 & 55.2 & 54.9 & 57.0 & 58.6 & 88.6 \\
\hline
0.0 & 0.4 & 57.3 & 56.7 & 59.7 & 60.8 & 91.2 \\
\hline
0.0 & 0.5 & 57.3 & 57.4 & 57.4 & 56.0 & 91.1 \\
\hline
0.1 & 0.1 & 58.2 & 55.7 & 59.5 & 62.5 & 85.5 \\
\hline
0.1 & 0.2 & 54.7 & 53.2 & 57.3 & 61.3 & 80.7 \\
\hline
0.1 & 0.3 & 54.3 & 53.9 & 55.1 & 56.9 & 79.2 \\
\hline
0.1 & 0.4 & 54.8 & 53.5 & 54.7 & 55.0 & 81.6 \\
\hline
0.1 & 0.5 & 54.9 & 54.3 & 53.8 & 60.4 & 79.2 \\
\hline
0.2 & 0.2 & 55.3 & 50.8 & 55.4 & 61.0 & 68.3 \\
\hline
0.2 & 0.3 & 54.2 & 51.6 & 55.6 & 55.4 & 71.1 \\
\hline
0.2 & 0.4 & 50.9 & 51.5 & 49.7 & 56.5 & 63.1 \\
\hline
0.2 & 0.5 & 50.4 & 50.7 & 52.5 & 54.9 & 59.6 \\
\hline
0.3 & 0.3 & 52.4 & 48.1 & 50.9 & 54.3 & 63.6 \\
\hline
0.3 & 0.4 & 52.2 & 51.7 & 51.2 & 54.0 & 62.4 \\
\hline
0.3 & 0.5 & 53.3 & 50.0 & 48.6 & 52.3 & 56.6 \\
\hline
0.4 & 0.4 & 51.0 & 52.4 & 53.1 & 50.0 & 53.7 \\
\hline
0.4 & 0.5 & 50.7 & 48.4 & 50.8 & 48.4 & 52.2 \\
\hline
0.5 & 0.5 & 49.1 & 49.7 & 49.0 & 48.8 & 48.5 \\
\hline
\end{tabular}
\end{minipage}

\subsection*{Experiment 3: Noise Dimensions}

\begin{minipage}{\textwidth}
\centering
\begin{tabular}{|c| c| c| c| c| c| c|}
\hline
\specialcell{XOR Dimensions\\(per source)} & \specialcell{Noise Dimensions\\(per source)} & KNN & Random Forest & SVM & Concatenate + GMKL & GMKL \\
\hline
3 & 0 & 86.7 & 80.0 & 84.3 & 97.5 & 99.8 \\
\hline
3 & 1 & 81.7 & 77.3 & 76.3 & 79.7 & 100.0 \\
\hline
3 & 2 & 76.0 & 78.2 & 74.3 & 74.5 & 98.9 \\
\hline
3 & 3 & 73.6 & 77.0 & 74.8 & 74.2 & 95.3 \\
\hline
3 & 6 & 61.7 & 69.6 & 69.4 & 74.4 & 77.0 \\
\hline
5 & 0 & 61.0 & 60.4 & 63.5 & 66.2 & 96.4 \\
\hline
5 & 1 & 58.4 & 56.6 & 61.4 & 66.7 & 72.3 \\
\hline
5 & 3 & 57.5 & 54.2 & 59.5 & 67.2 & 70.0 \\
\hline
5 & 5 & 57.0 & 54.2 & 58.4 & 66.3 & 65.6 \\
\hline
5 & 10 & 54.7 & 54.3 & 58.3 & 64.0 & 65.1 \\
\hline
7 & 0 & 57.7 & 54.9 & 60.2 & 63.5 & 74.9 \\
\hline
7 & 1 & 52.4 & 52.1 & 53.8 & 64.2 & 66.2 \\
\hline
7 & 4 & 55.3 & 53.8 & 56.5 & 59.8 & 62.1 \\
\hline
7 & 7 & 54.8 & 54.1 & 56.5 & 59.7 & 59.9 \\
\hline
7 & 14 & 51.6 & 52.1 & 54.5 & 61.9 & 59.4 \\
\hline
\end{tabular}
\end{minipage}










\newpage
\bibliographystyle{alpha}
\bibliography{mlsiml}

\end{document}

