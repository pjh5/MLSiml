\documentclass{article}
\usepackage{courier}
\renewcommand{\ttdefault}{pcr}
\usepackage[top=1in, bottom=1in, left=1in, right=1in]{geometry}
\usepackage{enumerate}
\usepackage{setspace}
\usepackage{amsmath}
\usepackage{fancyhdr}
\usepackage{graphicx}
\usepackage{listings}
\usepackage{booktabs}
\setlength{\parindent}{0cm}

\begin{document}
\title{Data Integration on High-Difficulty Binary Classification}
\author{Julia Finch, Jesse Hellemn, Kentaro Hoffman, Zhe Zhang}
\maketitle


\section{Overview and Motivation}
For many fields, especially in medicine and the social sciences, it is
increasingly common for data to come from multiple disparate data sources. For
example, a sociologist might be interested in predicting future income with two
datasets, family environment and school environment. For the best analysis, all
data sources should be utilized.

A naive approach is to concatenate the sources together into one large feature
matrix. Although simple and computationally efficient, this method throws away
information about the separate sources and forces downstream processing to
treat all of the sources in the same way. Various data integration techniques
have been proposed to more cleverly and effectively combine multiple sources.
Unfortunately, most of these techniques are poorly understood or poorly tested,
and there has been no systematic evaluation of their effectiveness.


\end{document}

